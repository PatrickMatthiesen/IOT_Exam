\section{Question 2 --- Wildlife Monitoring System}
\addcontentsline{toc}{section}{Question 2 --- Wildlife Monitoring System}

Building on the sensor device designed in Question 1, we now develop a monitoring system for tracking wildlife and vehicles in the two parks. This system will collect and analyze data from our hourly measurements to support wildlife research and park management.

\subsection{System Architecture and Data Flow}

The system uses a practical three-tier architecture that efficiently handles data from sensor devices. We want all the nodes to be independent and be able to just send data. So we want to have gateways placed strategically in the parks accounting for things like trees, buildings, and other obstacles that might block the signal. Each gateway will be connected to a LoRaWAN network server, which will then forward the data to central servers for storage and analysis. 

I am not sure what the best option would be for sending data from gateways to the central servers, but I will assume that we can use a cellular network, satellite uplink, or fixed wireless connection depending on the local infrastructure. I have been able to see on maps that each park has a few buildings around the parks that might be suitable for placing the gateways. Given the size of Knuthenborg we might only need 3-5 gateways, but given the bigger size of Mpala we might need more.

The idea of the central servers is to have a place to process and store the data collected from the sensor devices. With the servers having processed the data it will also be easier to create user applications that can visualize the data and provide insights for researchers and park managers. Data flowing in from the sensor devices will be stored in a time-series database, which is well-suited for handling the hourly data from multiple devices. The results from analytics on the servers will be stored in a separate relational database. The system architecture is illustrated in Figure\ref{fig:system_architecture}.   

\begin{figure}[h]
\centering
\begin{tikzpicture}[node distance=1.5cm]
% Define nodes - simplified version
\node (sensors) [draw, rectangle, minimum width=3cm, minimum height=1cm] {Sensor Devices (Animals \& Vehicles)};
\node (gateway) [draw, rectangle, below of=sensors, minimum width=3cm, minimum height=1cm] {LoRaWAN Gateway};
\node (lorawan) [draw, rectangle, below of=gateway, minimum width=3cm, minimum height=1cm] {LoRaWAN Network Server};
\node (servers) [draw, ellipse, below of=lorawan, minimum width=5cm, minimum height=2cm] {Main servers};
\node (logstorage) [draw, rectangle, below left of=servers, xshift=-1cm, yshift=-1cm, minimum width=2cm, minimum height=0.8cm] {Log Storage};
\node (processing) [draw, rectangle, below right of=servers, xshift=1cm, yshift=-1cm, minimum width=2cm, minimum height=0.8cm] {Analytics database};
\node (apps) [draw, rectangle, below of=processing, xshift=-2cm, minimum width=4cm, minimum height=0.8cm] {User Applications};

% Define edges
\draw[-stealth] (sensors) -- (gateway);
\draw[-stealth] (gateway) -- (lorawan);
\draw[-stealth] (lorawan) -- (servers);
\draw[-stealth] (servers) -- (logstorage);
\draw[-stealth] (servers) -- (processing);
\draw[-stealth] (logstorage) -- (apps);
\draw[-stealth] (processing) -- (apps);
\end{tikzpicture}
\caption{System Architecture for Wildlife Monitoring}
\label{fig:system_architecture}
\end{figure}

\subsubsection{Scalability and Flexibility}
Setting the system up in layers like this allows us to just add more devices, gateways and servers as needed. Given that each node only sends a message once per hour, the system should be able to handle a large number of devices without any issues. The bigger issue might come down to the gateways depending on how many devices we end up having to support. In a place like Mpala we might not be able to just add more gateways.
Servers on the other hand shouldn't be a problem as most applications for analysis would be able to scale horizontally. We can just add more servers to handle the increased load.

\subsection{Analytics and Data Processing}

The key parts of the system is the sensor devices and the servers that allow us to make data analysis on the data collected from the sensor devices. But we don't want to do all the processing on the servers, which would require us to send a lot of data which can be expensive for the devices. So we want to do some of the processing on the devices themselves, which will allow us to save power and bandwidth. 

The idea is to use TinyML models on the devices to do some basic processing of the data. The models will be trained to detect changes in the data, which will allow us to filter out meaningless data. For example we might not need to send the animals coordinates, if it has only moved up to 5 meters since the last measurement. This will allow us to save power and bandwidth, while still getting valuable insights from the data.

On the servers we will do more advanced processing of the combined data from all devices. The combination of data from multiple devices will allow us to do flock analysis, which will allow us to see how animals move in relation to each other. We can also do more advanced analysis of the particulate matter data, which will allow us to see how the air quality changes over time and how it relates to the animals movement.
The servers will also be able to draw in things like weather data to see how the weather affects things like the animals movement patterns and the air quality.

Interesting insights that we can get from the data include:

\begin{itemize}
  \item \textbf{Movement Patterns}: How far animals travel, areas they visit, and rest periods.
  \item \textbf{Territory Mapping}: Home ranges and territories for different species.
  \item \textbf{Seasonal Migration}: Patterns in movement across months and years.
  \item \textbf{Air Quality Maps}: Spatial maps of particulate matter concentration.
  \item \textbf{Vehicle Impact Assessment}: Correlation between vehicle locations and pollution patterns.
  \item \textbf{Health Indicators}: Activity levels and potential health issues based on movement patterns.
\end{itemize}

With the insights the staff in the parks will be able to see how the animals are doing and if there are any issues that need to be addressed. The insights can also be used to see how the air quality is doing in the parks, which can be used to see if there are any issues with pollution or other environmental factors.

Based on the insights we can decide on some ratings for the parks. We could use the data to build a health score based on movement patterns and air quality. The health score could be used to tell the park staff if they need to do a more thorough health check on specific animals or if they need to take action to improve the air quality in the park. 
We could build a risk assessment based on predictive health analytics. By analyzing historical data, we can identify patterns that precede health issues, allowing us to proactively monitor at-risk animals before they show symptoms.

\subsection{Conclusion}
The proposed monitoring system leverages hourly data collection to create a practical, energy-efficient solution for wildlife tracking. By focusing on what's achievable with periodic rather than continuous monitoring, we create a system that's both feasible to deploy in remote environments and capable of generating valuable insights for research and park management.

