\section{Q1 --- Networked Sensor Device for tracking}
\addcontentsline{toc}{section}{Q1 --- Networked Sensor Device for tracking}

We need a sensor device to track assets (vehicles and animals) in two wildlife parks:

\begin{itemize}
  \item \textbf{Knuthenborg Safari Park}, Denmark
  \item \textbf{Mpala Research Center}, Kenya
\end{itemize}

% - Knuthenborg: 1,600 acres, Denmark (Lat/Lon 54.811358, 11.5082552)
% - Mpala Research Center: 48,000 acres, Kenya (Lat/Lon 0.2882139, 36.9017593)
The parks are large, and the devices must be able to track the location of the assets with a high degree of accuracy. The devices will be used to monitor the animals' movements and behavior, as well as to track the vehicles used for research and conservation efforts.
The parks are located in different parts of the world, and the devices must be able to operate in different environments. The devices must be able to withstand harsh weather conditions, including extreme temperatures, rain, and dust. The devices must also be able to operate in remote areas with limited access to power and communication networks.  tun

\subsection{Functional requirements}
% The devices must be able to measure acceleration, particulate matter concentration, and track location. The devices must also be able to record data every hour and preferably report it in real-time.

\begin{itemize}
  \item Location tracking (GPS)
  \item Measure acceleration (motion state)
  \item Measure particulate matter concentration
  \item Record data every hour (preferably real-time reporting)
\end{itemize}

\subsection{Non-functional requirements}
The devices must have an accuracy of at least 500 meters. Apart from that we need to consider environmental conditions, battery life, size, and cost.

Given that the devices will be used in wildlife parks, they must be able to withstand extreme temperatures, rain, and dust. Knuthenborg Safari Park is located in Denmark, where temperatures in worst case can drop to -20 degrees Celsius in winter and have hot humid summers. Mpala Research Center is located in Kenya, where temperatures can reach close to 40 degrees Celsius in summer. 

Because the device needs to be able to track animals as many types of animals are present in the parks, the size of the device must be small enough to be attached to the animals without causing discomfort. 

\begin{itemize}
  \item Accuracy: $\geq$500 m or more
  \item Withstand extreme temperatures (e.g., -20 to 40 degrees Celsius)
  \item Withstand rain and dust (IP67 or better)
  \item Battery life: 0.5 years or more
  \item Track animals and vehicles
  \item Size has to be Small
  \item Cost: $\mathdollar100$ or less
\end{itemize}


\subsection{Device selection}

\subsection{Acceleration}
For the acceleration we can would want to use a accelerometer. An example of a sensor could be the ADXL345, which measures at 3mm x 5mm x 1mm, has a range of up to +/-16g, and can operate in temperatures from -40 to +85 degrees Celsius. ADXL345 can measure the acceleration of the device in three dimensions so vertical movement of birds can be detected too if needed. It has a startup of 1.4ms, with a resolution of 13 bits, typically operates at 2.5V, and has a draw of 23$\mu$A in normal operation and 0.1$\mu$A in standby mode\cite{adxl345}.

\subsection{Particulate matter concentration}
For the particulate matter concentration, we can use the Bosch BMV080 laser particle sensor\cite{bmv080}. This compact sensor measures only 4.2 mm x 3 mm x 3.5 mm, making it ideal for our size constraints. A key advantage of the BMV080 is its fan-less. Unlike traditional particulate matter sensors that require fans to draw air through the sensing chamber, the BMV080 is able to measure particulate matter concentration without a fan. This is would be beneficial for our use case, as it reduces the risk of dust buildup on the sensor. Dust is especially problematic in wildlife environments like the parks that we want to deploy the devices in, as it can clog the fan and affect the sensor's performance. The sensor operates with a supply voltage of 1.8 to 3.3 V and has a quick start-up time of 1.2 seconds. It can function in temperatures ranging from -15°C to +65°C, which covers our operational requirements. The BMV080 is has a sleep current of less than 3$\mu$A and an average total current of less than 68mA at 0.97Hz output data rate.
The datasheet conveniently also specifies that in duty cycling mode that we will use, then the sensor will use about 0.6mW each hour. 

%A restriction to keep in mind is the sensor requires a host that is ESP32, Cortex-M4 or Raspberry Pi Arm

\subsection{Location tracking}
For the location tracking, we can use a GPS module. The GPS module is a small device that provides real-time location data by using satellite signals to determine the device's position. It offers accurate tracking even in remote areas like Mpala. An example of such a module is the u-blox MIA-M10q, which measures just 4.5 mm x 4.5 mm x 1 mm\cite{u_blox_mia_m10q}. It supports our needs with a temperature range from -40°C to +85°C, has a cold start time of about 30 seconds, and has a positional accuracy of up to 2.5 meters. It has power consumption, drawing approximately 22mA at 3.3V during continuous tracking. However, with power supplied to the VBAT pin, it can retain satellite data and significantly reduce startup time to 1-10 seconds, using less energy per fix. In deep sleep mode with VBAT maintained, the module draws as little as 1$\mu$A.

\subsubsection{Microcontroller and communication}
Given that we want to send data over a long distance, we can use LoRaWAN for communication. An example of a microcontroller that supports LoRaWAN is the RAK3172 module based on the STM32WLE5 chip\cite{rak3172_datasheet}. The module has an ARM Cortex-M4 core, which fits the host requirement for the Bosch BMV080 sensor. The RAK3172 measuring only 15 mm x 15.5 mm x 3.5 mm, making it suitable for our small-size requirements.

The module supports a temperature range from -40 °C to +85 °C, which is ideal for the extreme conditions in both wildlife parks. It operates with a supply voltage of 2.0 to 3.6 V, compatible with our other components. The module provides multiple I/O ports including UART, SPI, ADC, GPIO and most notably I2C which many of the sensors require.

For communication, the RAK3172 supports frequency ranges from 150 MHz to 960 MHz with a bandwidth from 7.8 kHz to 500 kHz, appropriate for long-distance, low-power transmission required in large wildlife parks. It supports multiple regional bands including EU433, CN470, RU864, IN865, EU868, AU915, US915, KR920, and AS923, making it deployable in both Denmark and Kenya with the appropriate regional settings.

A plus for the RAK3172 is the low power consumption of just 1.69 $\mu$A in sleep mode, which is critical for our battery life requirements. In operation i would assume it would draw around 10 mA for processing and communication tasks, which is manageable within our power budget. The module includes 256 KB flash memory with ECC (Error Correction Code) and 64 KB RAM, providing sufficient storage for our sensor data and communication protocols.

For our application, we would use this microcontroller to collect data from the sensors (GPS, accelerometer, and particulate matter sensor), process the data, and transmit it over LoRaWAN to a base station.

As we want to at least be sure we can send data 1km, we will set the LoRa transmission power to 14 dBm, which is the maximum allowed in the EU868 band. This will allow us to achieve a range of up to 10 km in open areas, which is more than sufficient for our needs in both wildlife parks.

\subsubsection{Power supply}
The device will be powered by a rechargeable lithium-ion battery. The battery should be able to provide power for at least 0.5 years of operation.

To know how much power we need, we can calculate the power consumption of the device. This will be a naive calculation, as we have not yet tested real-world power consumption, but it will give us an idea of the power requirements.

We need to send data every hour, and the device will remain in sleep mode most of the time to conserve power. Our duty cycle strategy is as follows:

\begin{enumerate}
  \item \textbf{Wake-up phase}: The microcontroller wakes up from deep sleep at the beginning of each hour.
  \item \textbf{Sensor activation}: Sensors are activated sequentially to minimize peak current draw:
  \begin{itemize}
    \item Accelerometer activates first (1 second) due to its quick startup time
    \item Particulate matter sensor follows (1 second) 
    \item GPS module activates last (10 seconds) with warm start optimization when possible
  \end{itemize}
  \item \textbf{Data processing}: The microcontroller processes collected data (2 seconds)
  \item \textbf{Transmission}: Data is packaged and transmitted via LoRaWAN (1 second)
  \item \textbf{Sleep mode}: The device returns to deep sleep until the next hourly cycle
\end{enumerate}

This results in an active period of approximately 15 seconds per hour (0.42\% duty cycle), with the remaining 3,585 seconds in deep sleep mode.

Assuming the following power consumption:
\begin{itemize}
  \item GPS module: 22mA for 10 seconds every hour = 220mA·s
  \item Accelerometer: 23$\mu$A for 1 second every hour = 0.023mA·s
  \item Particulate matter sensor: 68mA for 1 second every hour = 68mA·s
  \item Microcontroller (active processing): 10mA for 20 seconds every hour = 200mA·s
  \item LoRaWAN transmission: 20mA for 1 second every hour = 20mA·s
  \item Sleep mode: 1.69$\mu$A for $(60*60) - 10 - 13 - 1 \simeq 3576$ seconds every hour = 6.044mA·s
\end{itemize}

Based on these values, we can calculate the total power consumption per hour.
We need to use the formula for power consumption $P = I \cdot t$, where $I$ is the current in milliamperes (mA) and $t$ is the time in seconds (s). The total power consumption per hour will be the sum of the power consumption of each component. The power consumption will be given in milliampere-seconds (mA·s) and then converted to milliampere-hours (mAh) by dividing by 3600 seconds per hour.

\begin{align*}
\text{Total per hour} &= 220\text{mA·s} + 0.023\text{mA·s} + 68\text{mA·s} + 130\text{mA·s} + 20\text{mA·s} + 6.044\text{mA·s} \\
&= 444.067\text{mA·s} \approx 0.1234\text{mAh per hour}
\end{align*}

For a 6-month (0.5 year) deployment, we can calculate the total battery capacity required:
\begin{align*}
\text{Total capacity} &= 0.1234\text{mAh} \times 24\text{h} \times 365.25\text{days} \times 0.5\text{year} \\
&= 0.1234\text{mAh} \times 24\text{h} \times 182.625\text{days} \\
&= 0.1234\text{mAh} \times 4,383\text{h} \\
&= 540.86\text{mAh}
\end{align*}

To account for battery self-discharge, temperature effects, and capacity degradation over time, we should include a safety factor of at least 1.5:
\begin{align*}
\text{Required capacity} &= 540.86\text{mAh} \times 1.5 \\
&= 811.29\text{mAh} \approx 800\text{mAh}
\end{align*}

Therefore, a 600mAh lithium-ion battery would be sufficient for our application. However, to provide additional reliability and to account for potential harsh operating conditions in wildlife environments, we would recommend using a 1000mAh battery, which provides extra margin while still maintaining a reasonable size and weight. % todo figure out why it is so low

For additional power supply, we can consider energy harvesting. 
A small solar panel rated at approximately 0.5W (measuring around 50mm × 50mm) would be sufficient to recharge the battery during daylight hours. With an average of just 1-2 hours of effective sunlight per day, this would provide enough additional energy to significantly extend the battery life beyond the required 6 months.

A potential issue would be that the device is not always exposed to sunlight, especially in the case of animals that might cover the cell from the sun with their body, fur or dirt. However, the generously-sized battery should compensate for periods without solar charging. The combination of a 1000mAh battery with solar recharging would likely enable the device to operate for well over a year in most deployment scenarios, and potentially indefinitely in locations with good sun exposure.

\subsection{Device enclosure}
To protect the device from environmental conditions, we can use an enclosure that is rated at least IP67. This means that the enclosure is dust-tight and can withstand immersion in water up to 1 meter for 30 minutes. The enclosure should also be able to withstand extreme temperatures, so we can use a material that can withstand temperatures from -40 to +85 degrees Celsius. We still need to allow air to get through the enclosure to allow the particulate matter sensor to do its work. A suitable material for the enclosure could be polycarbonate or ABS plastic, which are both durable and can withstand extreme temperatures. The enclosure should also be small enough to be attached to the animals without causing discomfort.

All the components we have selected would cover and area of about:

\begin{itemize}
  \item ADXL345: 3mm x 5mm x 1mm
  \item BMV080: 4.2mm x 3mm x 3.5mm
  \item u-blox MIA-M10: 4.5mm x 4.5mm x 1mm
  \item RAK3172: 15mm x 15.5mm x 3.5mm
  \item Battery: 30mm x 20mm x 5mm (assuming a small lithium-ion battery)
  \item Solar panel: 50mm x 50mm x 2mm (assuming a small solar panel)
\end{itemize}

The total area of the components would be approximately:
\begin{align*}
\text{Total area} &= (3 \times 5 \times 1) + (4.2 \times 3 \times 3.5) + (4.5 \times 4.5 \times 1) + (15 \times 15.5 \times 3.5) + (30 \times 20 \times 5) + (50 \times 50 \times 2)\\
&= 15 + 44.1 + 20.25 + 817.5 + 3000 + 5000 \\
&= 8856.85 \text{ mm}^3
\end{align*}

To account for the enclosure and additional components such as connectors lets say we triple the volume to allow for the enclosure and connectors, giving us a total volume of approximately 26,570 mm³ = 26.57 cm³.

\subsection{Payload Format and Transmission Strategy}
The device will transmit the following data in each hourly LoRaWAN payload:

\begin{itemize}
  \item Device ID (2 bytes): Unique identifier for each tracking device
  \item GPS Coordinates (8 bytes): Latitude and longitude (4 bytes each)
  \item Acceleration (6 bytes): X, Y, Z axis measurements (2 bytes each)
  \item Particulate Matter (2 bytes): PM2.5 concentration
  \item Battery Voltage (1 byte): For remote monitoring of device health
  \item Timestamp (4 bytes): UTC time of measurement
\end{itemize}

The total payload size is 23 bytes per transmission. This compact format balances information richness with bandwidth efficiency, critical for LoRaWAN networks where payload size directly impacts transmission time and power consumption. The small payload allows us to transmit using the highest data rate possible for the given range conditions, further optimizing power consumption during transmission.

For the network architecture, we would deploy LoRaWAN gateways at strategic locations throughout both wildlife parks. Given the size of Knuthenborg Safari Park (1,600 acres), approximately 2-3 gateways would provide sufficient coverage. For the much larger Mpala Research Center (48,000 acres), a network of 10-15 gateways positioned at elevated locations would be required to ensure continuous connectivity.

\subsection{Design Trade-offs}
The design balances several competing factors:

\begin{itemize}
  \item \textbf{Energy vs Data Frequency}: Hourly measurements provide sufficient temporal resolution for tracking wildlife movement patterns while keeping power consumption low. More frequent transmissions would provide better real-time tracking but would significantly reduce battery life. This hourly interval offers a good compromise for monitoring both fast-moving vehicles and slower wildlife migration patterns.
  
  \item \textbf{Size vs Battery Capacity}: The 1000mAh battery offers sufficient capacity but increases device size. For smaller animals, a 600mAh option could be considered to reduce weight at the expense of operational lifetime. We prioritized having a safety margin in our power budget to accommodate unexpected environmental conditions.
  
  \item \textbf{Network Selection}: We chose LoRaWAN over alternative technologies for several reasons:
  \begin{itemize}
    \item Compared to cellular (GSM/LTE): LoRaWAN offers significantly lower power consumption and better coverage in remote areas, though with lower data rates.
    \item Compared to satellite: LoRaWAN provides much lower cost per device and lower power consumption, though with more limited coverage requiring gateway infrastructure.
    \item Compared to Bluetooth/WiFi: LoRaWAN offers drastically increased range at the expense of bandwidth.
  \end{itemize}
  
  \item \textbf{Transmission Power vs Range}: Operating at 14 dBm maximizes coverage range but increases transmission power consumption. In areas with dense gateway coverage, adaptive power settings could be implemented to reduce transmission power when in proximity to gateways.
  
  \item \textbf{Sensor Selection}: The BMV080 particulate sensor offers fan-less operation ideal for dusty environments but has a higher temperature minimum (-15°C) than other components. This represents a compromise between reliability and temperature range, slightly limiting operation in extremely cold conditions at Knuthenborg during winter.
  
  \item \textbf{Solar Power vs Battery Only}: Adding a solar panel increases the device size but provides potentially unlimited operational lifetime in sunny conditions. For heavily forested areas or for animals with nocturnal habits, the battery capacity alone ensures minimum operational requirements are met, while solar charging provides extended operation in favorable conditions.
\end{itemize}

The final design prioritizes reliability, battery life, and size constraints while ensuring data quality sufficient for wildlife monitoring applications in both park environments. The architecture optimizes for extremely low duty cycle operation, with the device spending over 99.5% of its time in deep sleep mode to maximize battery life.